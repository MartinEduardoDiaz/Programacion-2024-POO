% Tipo Presentación
\documentclass{beamer}

% Paquetes
\usepackage[utf8]{inputenc}
\usepackage[spanish]{babel}
\usepackage{graphicx}
\usepackage{hyperref}
\hypersetup{colorlinks=true, linkcolor=blue, urlcolor=blue}
\usepackage{amsmath}
\usepackage{listings}
\usepackage{booktabs}
\usepackage{listing}

% Tema
\usetheme{Madrid}

\title{Introducción a LaTeX}
\subtitle{Presentación}
\author{Martin Diaz}
\date{\today}


\begin{document}
	
	\begin{frame}
		\titlepage
	\end{frame}
	
	\begin{frame}
		\begin{itemize}
			\item a
			\item e
			\item i
			\item o
			\item u
			\item Link \url{http://www.google.com/ }
		\end{itemize}
	\end{frame}
	
	\begin{frame}[fragile]
		\frametitle{Ejemplo de muestra de Código}
		\begin{lstlisting}[language=TeX, frame=single]
	\documentclass{article}
		\begin{document}
			Contenido
	\end{document}
		\end{lstlisting}
	\end{frame}
	
	\begin{frame}(Formulas Matemáticas Complejas)
		\begin{equation*}
			\int_{-\infty}^\infty e^{-x^2} dx =\sqrt{\pi}
		\end{equation*}
	\end{frame}

\end{document}